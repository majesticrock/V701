\section{Zielsetzung}
    In diesem Versuch soll die Reichweite von $\alpha$-Strahlung bestimmt werden, sowie die Statistik des radioaktiven Zerfalls überprüft werden.
\section{Theorie}
\label{sec:Theorie}
    Durch Bestimmung der Reichweite von $\alpha$-Strahlung kann auf die zugehörige Energie geschlossen werden. Wenn $\alpha$-Strahlung
    Materie durchläuft kann diese einerseits durch elastische Stöße mit dem Material Energie verlieren, was hier aber eine untergeordnete
    Rolle spielt, andererseits durch Anregung beziehungsweise Dissoziation von Molekülen Energie abgeben. Der Energieverlust
    $\frac{\symup{d} \, E_{\symup{\alpha}}} {\symup{d} \, x}$ lässt sich durch die Bethe-Bloch Gleichung ausdrücken:
    \begin{equation}
    \label{eqn:bethe-bloch}
       - \frac{\symup{d} \, E_{\symup{\alpha}}} {\symup{d} \, x} = \frac{q^2 e^4} {4 \pi \epsilon_{\symup{0}} m_{\symup{e}}} \frac{n Z} {v^2} \ln \Bigl( \frac{2 m_{\symup{e}} v^2 }{I} \Bigr).
   \end{equation}    
   Hierbei ist $q$ die Ladung und $v$ die Geschwindigkeit der $\alpha$-Strahlung; $Z$ die Ordnungszahl, $n$ die Teilchendichte und $I$ die 
   Ionisierungsenergie des beschossenen Gases. Die Reichweite $R$ lässt sich über 
   \begin{equation}
   \label{eqn:Reichweite}
        R = \int_0^{E_{\symup{\alpha}}} \frac{\symup{d} \, E_{\symup{\alpha}}} {-\symup{d} \, E_{\symup{\alpha}} / \symup{d} \, x  }
   \end{equation}
   berechnen. Die Bethe-Bloch Gleichung besitzt allerdings nur eine Gültigkeit für hohe Energien. Für niedrigere Energien kann die mittlere
   Reichweite $R_{\symup{m}}$, also die Reichweite, die die Hälfte der $\alpha$-Teilchen noch erreichen, über experimentell gewonnene Kurven bestimmt werden.
   Für $\alpha$-Strahlung in Luft mit Energien von $E_{\symup{\alpha}} \leq 2.5 \symup{MeV}$ gilt 
   \begin{equation}
   \label{eqn:mittlere-reichweite}
        R_{\symup{m}} = 3.1 \cdot E_{\symup{\alpha}}^{\frac{3}{2}}, 
   \end{equation}     
   wobei $ R_{\symup{m}} $ hier die Einheit mm hat und $E_{\symup{\alpha}}$ in MeV vorliegt.
   Zur Bestimmung der Reichweite von $\alpha$-Strahlung kann eine Absorptionsmessung durchgeführt werden, wobei der Umgebungsdruck variiert wird,
   da die Reichweite von $\alpha$-Strahlung bei konstantem Druck und konstanter Temperatur, proportional zum Druck $p$ ist. Für einen 
   festen Abstand $x_{\symup{0}}$ zwischen Detektor und Strahler lässt sich die effektive Länge $x$ über 
   \begin{equation}
   \label{eqn:effektive-laenge}
        x = x_{\symup{0}} \frac{p}{p_{\symup{0}}}
   \end{equation}
   bestimmen, wobei $p_{\symup{0}}$ der Normaldruck ist.     