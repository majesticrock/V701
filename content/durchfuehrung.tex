\section{Durchführung}
\label{sec:Durchführung}
    Bei diesem Versuch befindet sich ein Americium-Präparat in einem abgeschlossenen Glaszylinder vor einem Halbleiter-Detektor.
    Im Halbleiterdetektor enstehen beim Auftreffen der Strahlung freie Ladungsträger welche einen elektrischen Strom zu einem Vorverstärker 
    bilden. Das Sigal wird anschließend von einem Vielkanalanalysator analysiert, welcher an einen Computer angeschlossen ist. Der Abstand
    zwischen Detektor und Präparat ist einstellbar. Des Weiteren ist der Glaszylinder an einer Vakuumpumpe angeschlossen, womit der Druck
    innerhalb des Zylinders variabel ist. Der Vielkanalanalysator analysiert die Pulse bezüglich ihrer Höhe, wobei die Pulshöhe proportional zur Energie
    des $\alpha$-Teilchens ist. Die Auswertung erfolgt schließlich durch ein Computerprogramm.
    Vor der Durchführung der einzelnen Versuchsteilse wird zunächst die korrekte Verkabelung des Versuchsaufbaues überprüft.
    Anschließend werden grundlegende Einstellungen im Computerprogramm getätigt.
    Zur Auswertung wird das Programm \emph{Multichannel Analyzer} verwendet. Dieses muss mit dem Vielkanalanalysator verbunden sein, damit der 
    Schalter unter \emph{MCA STATUS} auf \emph{connectet} gestellt werden kann. Die Messzeit kann, wenn der Schalter bei \emph{Acquisition} auf \emph{AUTO} steht, bei 
    \emph{Measurement time} eingestellt werden. Im Zustand \emph{MANUEL} muss die Messung durch den Benutzer selbst beendet werden.
    Die Gesamtzählrate ist bei den Messungen jeweils unter dem Punkt \emph{pulses detected} aufgeführt.
    \subsection{Bestimmung der Reichweite}
        Zunächst muss die Diskriminatorschwelle am Vielkanalanalysator eingestellt werden. Hierzu wird der Abstand zwischen Detektor und 
        Präparat auf ein Maximum gestellt und so weit verringert, dass der Detektor gerade Pulse unter \emph{pulses detected} aufnimmt. Nun wird der Druck im Zylinder
        auf ca. $0$ mbar verringert. Die Messung erfolgt bei konstantem Abstand für verschiedene Drücke im Abstand von $50$ mbar je zwei Minuten 
        lang. Wenn die gemessenen Pulse stark abfallen, wird der Messabstand auf $25$ mbar verfeinert. Aufgenommen werden bei jeder Messung
        die Gesamtzählrate, sowie der Kanal, die Position, bei dem das Energiemaximum auftritt. 
        Anschließend wird diese Messung für einen weiteren Abstand durchgeführt. 
    \subsection{Statistik des radioaktiven Zerfalls}    
        Für diesen Versuchsteil wird der selbe Versuchsaufbau wie zuvor verwendet. Der Druck im Glaszylinder wird auf $0$ mbar abgesenkt und über
        die gesamte Messung konstant gehalten. Bei dieser Messung wird jeweils $10$ Sekunden lang die Anzahl der Zefälle aufgenommen und diese Messung
        insgesamt 100 mal wiederholt.