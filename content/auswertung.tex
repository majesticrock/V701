\section{Auswertung}
\label{sec:Auswertung}

\subsection{Distanz von 2,4 cm}

Zur Bestimmung der mittleren Reichweite der $\alpha$-Strahlung werden die Daten aus \autoref{tab:d24cm} geplottet.
Dazu werden die Counts gegen effektive Länge $x := \dfrac{x_0 p}{p_0}$ aufgetragen.
Des Weiteren wird durch den starken nahezu linearen Abfall der Counts gegen Ende der Messung ein linearer Fit der Form

\begin{equation}
    c(x) = a_{2,4} \cdot x + b_{2,4}
\end{equation}

gelegt.

\begin{table}[!htp]
\centering
\caption{Messwerte bei $x_0 = 2,4$ cm.}
\label{tab:d24cm}
\begin{tabular}{S[table-format=4.0] S[table-format=5.0] S[table-format=4.0] S[table-format=1.2]}
\toprule
{$p$ / mbar} & {Count} & {Channel} & {$E_{2,4}$ / MeV} \\
\midrule
   0 & 67653 & 1251 & 4.00 \\
  50 & 66720 & 1248 & 3.99 \\
 100 & 66429 & 1200 & 3.83 \\
 150 & 66413 & 1167 & 3.73 \\
 200 & 65531 & 1152 & 3.68 \\
 250 & 65111 & 1068 & 3.41 \\
 300 & 65124 & 1023 & 3.27 \\
 350 & 63854 &  975 & 3.11 \\
 400 & 63410 &  967 & 3.09 \\
 450 & 61449 &  926 & 2.96 \\
 500 & 61302 &  896 & 2.86 \\
 550 & 58850 &  847 & 2.70 \\
 600 & 57827 &  799 & 2.55 \\
 650 & 53067 &  743 & 2.37 \\
 700 & 47575 &  704 & 2.25 \\
 725 & 41566 &  687 & 2.19 \\
 750 & 35397 &  655 & 2.09 \\
 775 & 29017 &  640 & 2.04 \\
 800 & 23405 &  632 & 2.04 \\
 825 & 15651 &  634 & 2.03 \\
 850 &  9099 &  634 & 2.02 \\
 875 &  4997 &  632 & 2.02 \\
 900 &  2708 &  631 & 2.02 \\
 925 &  1119 &  632 & 2.02 \\
 950 &   382 &  638 & 2.02 \\
 975 &    72 &  640 & 2.02 \\
1000 &    35 &  634 & 2.01 \\
\bottomrule
\end{tabular}
\end{table}

\begin{figure}
    \centering
    \includegraphics[width=0.9\textwidth]{build/plot_countd24.pdf}
    \caption{Plot der Messwerte bei $x_0 = 2,4$ cm, wobei über den starken Abfall ein linearer Fit gelegt ist.}
    \label{fig:plot24}
\end{figure}

Dieser Plot ist in \autoref{fig:plot24} zu sehen.
Mittels Python 3.7.0 werden die Koeffizienten wie folgt bestimmt:

\begin{center}
    $a_{2,4} = (-9,0 \pm 0,6) \cdot 10^{4}$ $\frac{1}{\symup{cm}}$

    $b_{2,4} = (2,0 \pm 0,1) \cdot 10^{5}$.
\end{center}

Daraus errechnet sich direkt die mittlere Reichweite 

\begin{center}
    $x_\text{mittel} = \frac{c_{2,4} - b_{2,4}}{a_{2,4}} = (1,5 \pm 0,2)$ cm.
\end{center}

wobei $c_{2,4}$ der höchste "Count"-Wert aus \autoref{tab:d24cm} ist. Der zugehörige Fehler nach Gauß berechnet sich nach

\begin{equation}
    \Delta x_\text{mittel} = \sqrt{\bigg(\frac{1}{a_{2,4}} \cdot \Delta b_{2,4} \bigg)^2 + \bigg( \frac{c_{2,4}-b_{2,4}}{a_{2,4}^2} \cdot \Delta a_{2,4} \bigg)^2}.
\end{equation}

Um die Energie der Strahlung an dieser Stelle zu errechnen werden die Energiewerte aus \autoref{tab:d24cm} gegen die effektive Länge in einem Plot aufgetragen.
Durch die Werte, die sich linear verhalten, wird ein Fit der Form

\begin{equation}
    E_{2,4}(x) = m_{2,4} x + n_{2,4}
\end{equation}

gelegt. Dieser Plot ist in \autoref{fig:plotE24} zu sehen

\begin{figure}
    \centering
    \includegraphics[width=0.9\textwidth]{build/plot_E24.pdf}
    \caption{Plot der Energien bei $x_0 = 2,4$ cm, wobei über den linearen Teil ein Fit gelegt ist.}
    \label{fig:plotE24}
\end{figure}

Die Koeffizienten werden erneut mittels Python 3.7.0 bestimmt. Sie betragen

\begin{center}
    $m_{2,4} = (-1,09 \pm 0,02)$ $\frac{\symup{MeV}}{\symup{cm}}$

    $n_{2,4} = (4,09 \pm 0,2)$ MeV.
\end{center}

Damit lässt sich $E_{2,4}(x_\text{mittel}) = (2,5 \pm 0,3)$ MeV berechnen. Der Fehler berechnet sich dabei nach

\begin{equation}
    \Delta E(x_\text{mittel}) = \sqrt{\bigg( x_\text{mittel} \cdot m_{2,4} \Delta \bigg)^2 + \bigg(m_{2,4} \cdot \Delta x_\text{mittel} \bigg)^2 + \bigg( \Delta n_{2,4} \bigg)^2}.
\end{equation}



\subsection{Distanz von 4,2 cm}

Die gleiche Prozedur wird nun mit einem anderen $x_0$ durchgeführt. Dieses beträgt nun $4,2$ cm.
Zunächst werden aus den Daten in \autoref{tab:d42cm} die Counts gegen die effektive Länge aufgetragen. Erneut wird ein Fit der Form

\begin{equation}
    c(x) = a_{4,2} \cdot x + b_{4,2}
\end{equation}

über den linearen Abfall gelegt. Der entstehende Plot ist in \autoref{fig:plot42} zu sehen.

\begin{table}[!htp]
\centering
\caption{Messwerte bei $x_0 = 4,2$ cm.}
\label{tab:d42cm}
\begin{tabular}{S[table-format=3.0] S[table-format=5.0] S[table-format=4.0] S[table-format=1.2]}
\toprule
{$p$ / mbar} & {Count} & {Channel} & {$E_{4,2}$ / MeV} \\
\midrule
  0 & 28410 & 1276 & 4.07 \\
 50 & 27626 & 1216 & 3.88 \\
100 & 27574 & 1103 & 3.52 \\
150 & 26984 & 1017 & 3.25 \\
200 & 26657 &  983 & 3.14 \\
250 & 25812 &  896 & 2.96 \\
300 & 25006 &  928 & 2.86 \\
350 & 23405 &  835 & 2.66 \\
400 & 20621 &  736 & 2.35 \\
450 & 15123 &  655 & 2.09 \\
475 & 11615 &  649 & 2.07 \\
500 &  7059 &  640 & 2.06 \\
525 &  2627 &  632 & 2.04 \\
550 &   991 &  633 & 2.02 \\
575 &   199 &  645 & 2.02 \\
600 &     3 &  629 & 2.01 \\
625 &     0 &    0 & 0 \\
\bottomrule
\end{tabular}
\end{table}

\begin{figure}
    \centering
    \includegraphics[width=0.9\textwidth]{build/plot_countd42.pdf}
    \caption{Plot der Messwerte bei $x_0 = 4,2$ cm, wobei über den starken Abfall ein linearer Fit gelegt ist.}
    \label{fig:plot42}
\end{figure}

Mittels Python 3.7.0 werden die Koeffizienten bestimmt:

\begin{center}
    $a_{4,2cm} = (-2,8 \pm 0,2) \cdot 10^{4}$ $\frac{1}{\symup{cm}}$

    $b_{4,2cm} = (6,6 \pm 0,4) \cdot 10^{4}$.
\end{center}

Damit kann erneut die mittlere Reichweite bestimmt werden:

\begin{center}
    $x_\text{mittel} = \frac{c_{4,2} - b_{4,2}}{a_{4,2}} = (1,3 \pm 0,2)$ cm.
\end{center}

Dabei ist $c_{4,2}$ der höchste "Count"-Wert aus \autoref{tab:d42cm}. Der Fehler nach  Gauß berechnet sich nach

\begin{equation}
    \Delta x_\text{mittel} = \sqrt{\bigg(\frac{1}{a_{4,2}} \cdot \Delta b_{4,2} \bigg)^2 + \bigg( \frac{c_{4,2}-b_{4,2}}{a_{4,2}^2} \cdot \Delta a_{4,2} \bigg)^2}.
\end{equation}

Zur Bestimmung der Energie werden die Werte aus \autoref{tab:d42cm} aufgetragen. Über das lineare Gefälle wird ein Fit der Form

\begin{equation}
    E_{4,2}(x) = m_{4,2} x + n_{4,2}
\end{equation}

gelegt. Der entstehende Plot ist in \autoref{fig:plotE24} zu sehen.

\begin{figure}
    \centering
    \includegraphics[width=0.9\textwidth]{build/plot_E24.pdf}
    \caption{Plot der Energien bei $x_0 = 4,2$ cm, wobei über den linearen Teil ein Fit gelegt ist.}
    \label{fig:plotE24}
\end{figure}

In diesem Plot ist aus Gründen der Lesbarkeit und Skalierung der letzte Wert nicht aufgenommen. Bei diesem wurden keine Strahlung mehr gemessen.
Mittels Python 3.7.0 werden die Koeffizienten als

\begin{center}
    $m_{4,2} = (-0,85 \pm 0,04)$ $\frac{\symup{MeV}}{\symup{cm}}$

    $n_{4,2} = (3,91 \pm 0,07)$ MeV
\end{center}

bestimmt.
Somit wird $E_{4,2}(x_\text{mittel}) = (2,8 \pm 0,2)$ MeV berechnet. Der zugehörige Fehler berechnet sich nach

\begin{equation}
    \Delta E(x_\text{mittel}) = \sqrt{\bigg( x_\text{mittel} \cdot \Delta m_{4,2} \bigg)^2 + \bigg(m_{4,2} \cdot \Delta x_\text{mittel} \bigg)^2 + \bigg(\Delta n_{4,2} \bigg)^2}.
\end{equation}




\subsection{Verteilung der Anzahl der "Counts"}

Die aufgenommen Messwerte bei einer Integrationszeit von $\Delta t = 10$ s sind in \autoref{tab:100werte} zu sehen.

\begin{table}[!htp]
\centering
\caption{Anzahl von Counts bei 10s Integrationszeit.}
\label{tab:100werte}
\begin{tabular}{S[table-format=4.0] S[table-format=4.0] S[table-format=4.0] S[table-format=4.0] S[table-format=4.0] S[table-format=4.0] S[table-format=4.0] S[table-format=4.0] S[table-format=4.0] S[table-format=4.0]}
\toprule
\multicolumn{10}{c}{Counts} \\
\midrule

3749 & 4062 & 3812 & 3971 & 3502 & 3827 & 3879 & 3743 & 4069 & 3811 \\
3701 & 3549 & 3773 & 3980 & 3867 & 4014 & 3728 & 3849 & 3636 & 3688 \\
3884 & 4099 & 3744 & 3871 & 3810 & 3948 & 3702 & 3610 & 3864 & 3839 \\
3939 & 3808 & 3628 & 3962 & 3792 & 3882 & 3678 & 3593 & 3725 & 3794 \\
3635 & 3903 & 3651 & 4070 & 4060 & 3936 & 4055 & 3804 & 3776 & 3725 \\
3865 & 3997 & 3809 & 3694 & 3576 & 3670 & 3607 & 3832 & 3935 & 3770 \\
3602 & 3813 & 3841 & 3897 & 3951 & 3901 & 3976 & 3722 & 3755 & 3806 \\
3911 & 3580 & 4037 & 3805 & 3989 & 3957 & 3951 & 4013 & 3751 & 3932 \\
3848 & 3981 & 3776 & 3972 & 3911 & 3954 & 3637 & 3567 & 3942 & 3760 \\
3904 & 4063 & 3706 & 3697 & 3973 & 3910 & 3733 & 3870 & 3741 & 3856 \\ 

\bottomrule
\end{tabular}
\end{table}

Der Erwartungswert dieser Daten berechnet sich nach

\begin{equation}
    \mu = \sum_{i=1}^{N} x_i p_i,
\end{equation}

wobei $N$ die Anzahl der Messwerte, $x_i$ der jeweilige Wert und $p_i$ die Wahrscheinlichkeit für den jeweiligen Wert ist.
In diesem Fall ist $N = 100$ und $p_i = \dfrac{1}{N}$ für alle Daten.

Damit beträgt $\mu \approx 3827,43$.
Damit lässt sich die Varianz wie folgt berechnen:

\begin{equation}
    \sigma^2 = \sum_{i=1}^{N} (x_i - \mu)^2 \cdot p_i \approx 19418,025.
\end{equation}

Damit lässt sich eine Gaußkurve der Form

\begin{equation}
    f(x | \mu, \sigma^2) = \frac{1}{\sqrt{2 \pi \sigma^2}} \cdot e^{- \frac{(x_i - \mu)^2}{2 \sigma^2}}
\end{equation}

berechnen. Zum Vergleich der Werte mit dieser Kurve werden die Werte in einem Balkendiagramm zusammen mit der Kurve aufgetragen.
Die Kurve wird so normiert, dass ihr Peak genauso hoch ist, wie der höchste Balken. Der so entstehende Plot ist in \autoref{fig:plot_gauss} zu sehen.

\begin{figure}
    \centering
    \includegraphics[width=0.9\textwidth]{build/plot_gauss.pdf}
    \caption{Balkendiagramm der Count-Messwerte mit einer Balkenbreite von 66 Counts und einer Gaußverteilung.}
    \label{fig:plot_gauss}
\end{figure}

Des Weiteren wird ein Plot erstellt, in dem die Daten mit einer Poissonverteilung verglichen werden.

Eine Poissonverteilung berechnet sich nach

\begin{equation}
    P_\lambda (k) = \frac{\lambda^k}{k!} e^{-\lambda},
\end{equation}

wobei $k \in \mathds{N}$ und einem $\lambda$, das die Varianz und den Erwartungswert beschreibt. 
Da für die Poissonverteilung nur diskrete $k$ benutzt werden können und $k=0$ dem kleinsten Wert entsprechen muss,
muss $\lambda$ entsprechend normiert werden.
Es berechnet sich also über

\begin{equation}
    \lambda = \sum_{i=1}^{N} (p_i (x_i - x_min) \cdot \frac{1}{L_\text{bin}}) \approx 4,9.
\end{equation}

Dabei ist $L_\text{bin} = 66$ die Größe der einzelnen Bins.
Die Balken der so entstehenden Poissonverteilung werden so normiert, dass der größte dieser denselben Wert wie der größte der Messwerte besitzt.
Der entsprechende Plot ist in \autoref{fig:plot_poisson} zu finden.

\begin{figure}
    \centering
    \includegraphics[width=0.9\textwidth]{build/plot_poisson.pdf}
    \caption{Balkendiagramm der Count-Messwerte mit einer Balkenbreite von 66 Counts und einer Poissonverteilung.}
    \label{fig:plot_poisson}
\end{figure}