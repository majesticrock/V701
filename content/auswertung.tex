\section{Auswertung}
\label{sec:Auswertung}

In diesem Kapitel werden die zunächst die Messwerte betrachtet, die bei einer Distanz $x_0 = 2,4$ cm entstanden sind, betrachtet.
Anschließend werden die Werte bei $x_0 = 4,2$ cm ausgewertet und zum Schluss wird die statistische Verteilung der Zerfälle überprüft.

\subsection{Auswertung der Werte bei einer Distanz von 2,4 cm}

Zur Bestimmung der mittleren Reichweite der $\alpha$-Strahlung werden die Daten aus \autoref{tab:d24cm} geplottet.
Dazu wird die Gesamtzählrate gegen die effektive Länge $x := \dfrac{x_0 p}{p_0}$ aufgetragen.
Des Weiteren wird durch den starken nahezu linearen Abfall der Gesamtzählrate bei den Werte 14 bis 23 (650 mbar bis 875 mbar) ein linearer Fit der Form

\begin{equation}
    c(x) = a_{2,4} \cdot x + b_{2,4}
\end{equation}

gelegt.

\begin{table}[!htp]
\centering
\caption{Messwerte bei $x_0 = 2,4$ cm.}
\label{tab:d24cm}
\begin{tabular}{S[table-format=4.0] S[table-format=5.0] S[table-format=4.0] S[table-format=1.2]}
\toprule
{$p$ / mbar} & {Count} & {Channel} & {$E_{2,4}$ / MeV} \\
\midrule
   0 & 67653 & 1251 & 4.00 \\
  50 & 66720 & 1248 & 3.99 \\
 100 & 66429 & 1200 & 3.83 \\
 150 & 66413 & 1167 & 3.73 \\
 200 & 65531 & 1152 & 3.68 \\
 250 & 65111 & 1068 & 3.41 \\
 300 & 65124 & 1023 & 3.27 \\
 350 & 63854 &  975 & 3.11 \\
 400 & 63410 &  967 & 3.09 \\
 450 & 61449 &  926 & 2.96 \\
 500 & 61302 &  896 & 2.86 \\
 550 & 58850 &  847 & 2.70 \\
 600 & 57827 &  799 & 2.55 \\
 650 & 53067 &  743 & 2.37 \\
 700 & 47575 &  704 & 2.25 \\
 725 & 41566 &  687 & 2.19 \\
 750 & 35397 &  655 & 2.09 \\
 775 & 29017 &  640 & 2.04 \\
 800 & 23405 &  632 & 2.04 \\
 825 & 15651 &  634 & 2.03 \\
 850 &  9099 &  634 & 2.02 \\
 875 &  4997 &  632 & 2.02 \\
 900 &  2708 &  631 & 2.02 \\
 925 &  1119 &  632 & 2.02 \\
 950 &   382 &  638 & 2.02 \\
 975 &    72 &  640 & 2.02 \\
1000 &    35 &  634 & 2.01 \\
\bottomrule
\end{tabular}
\end{table}

\begin{figure}
    \centering
    \includegraphics[width=0.9\textwidth]{build/plot_countd24.pdf}
    \caption{Plot der Gesamtzählrate gegen den Druck bei $x_0 = 2,4$ cm, wobei über den starken Abfall ein linearer Fit gelegt ist.}
    \label{fig:plot24}
\end{figure}

Dieser Plot ist in \autoref{fig:plot24} zu sehen.
Mittels Python 3.7.0 werden die Koeffizienten wie folgt bestimmt:

\begin{align}
    a_{2,4} &= (-9,0 \pm 0,6) \cdot 10^{4} \frac{1}{\symup{cm}},\\
    b_{2,4} &= (2,0 \pm 0,1) \cdot 10^{5}.
\end{align}

Daraus errechnet sich direkt die mittlere Reichweite 

\begin{center}
    $R_\text{m} = \frac{c_{2,4} - b_{2,4}}{a_{2,4}} = (1,9 \pm 0,2)$ cm.
\end{center}

wobei $c_{2,4}$ die Hälfte der höchsten Gesamtzählrate aus \autoref{tab:d24cm} ist. Der zugehörige Fehler nach Gauß berechnet sich nach

\begin{equation}
    \Delta R_\text{m} = \sqrt{\bigg(\frac{1}{a_{2,4}} \cdot \Delta b_{2,4} \bigg)^2 + \bigg( \frac{c_{2,4}-b_{2,4}}{a_{2,4}^2} \cdot \Delta a_{2,4} \bigg)^2}.
\end{equation}

Um die Energie der Strahlung an dieser Stelle zu errechnen, werden die Energiewerte aus \autoref{tab:d24cm} gegen die effektive Länge in einem Plot aufgetragen.
Durch die Werte, die sich linear verhalten, wird ein Fit der Form

\begin{equation}
    E_{2,4}(x) = m_{2,4} x + n_{2,4}
\end{equation}

gelegt. Dieser Plot ist in \autoref{fig:plotE24} zu sehen

\begin{figure}
    \centering
    \includegraphics[width=0.9\textwidth]{build/plot_E24.pdf}
    \caption{Plot der Energien bei $x_0 = 2,4$ cm, wobei über den linearen Teil ein Fit gelegt ist.}
    \label{fig:plotE24}
\end{figure}

Die Koeffizienten werden erneut mittels Python 3.7.0 bestimmt. Sie betragen

\begin{align}
    m_{2,4} &= (-1,09 \pm 0,02) \frac{\symup{MeV}}{\symup{cm}}\\
    n_{2,4} &= (4,09 \pm 0,2) \symup{MeV}.
\end{align}

Damit lässt sich $E_{2,4}(R_\text{m}) = (2,1 \pm 0,2)$ MeV berechnen. Der Fehler berechnet sich dabei nach

\begin{equation}
    \Delta E(R_\text{m}) = \sqrt{\bigg( R_\text{m} \cdot m_{2,4} \Delta \bigg)^2 + \bigg(m_{2,4} \cdot \Delta R_\text{m} \bigg)^2 + \bigg( \Delta n_{2,4} \bigg)^2}.
\end{equation}

Des Weiteren wird zu Vergleichszwecken die Energie nach \eqref{eqn:mittlere-reichweite} berechnet:

\begin{center}
    $E_{2,4,R} = \sqrt[3]{\frac{R_\text{m}^2}{3,1}} = (3,2 \pm 0,2)$ MeV,
\end{center}

wobei sich der Fehler nach Gauß über

\begin{equation}
\label{eqn:fehlerenergie}
    \Delta E_{R,2,4} = \sqrt{ \bigg ( \dfrac{2\sqrt[3]{100}}{3\sqrt[3]{961 R_\text{m}}} \cdot \Delta R_\text{m} \bigg)^2 }
\end{equation}

berechnet.

\subsection{Auswertung der Werte bei einer Distanz von 4,2 cm}

Die gleiche Prozedur wird nun mit einer anderen Distanz $x_0$, also einer anderen effektiven Länge pro Druck, durchgeführt. Dieses beträgt nun $4,2$ cm.
Zunächst werden aus den Daten in \autoref{tab:d42cm} die Gesamtzählrate gegen die effektive Länge aufgetragen. Erneut wird ein Fit der Form

\begin{equation}
    c(x) = a_{4,2} \cdot x + b_{4,2}
\end{equation}

über den linearen Abfall bei den Werten 6 bis 14 (250 mbar bis 525 mbar) gelegt. Der entstehende Plot ist in \autoref{fig:plot42} zu sehen.

\begin{table}[!htp]
\centering
\caption{Messwerte bei $x_0 = 4,2$ cm.}
\label{tab:d42cm}
\begin{tabular}{S[table-format=3.0] S[table-format=5.0] S[table-format=4.0] S[table-format=1.2]}
\toprule
{$p$ / mbar} & {Count} & {Channel} & {$E_{4,2}$ / MeV} \\
\midrule
  0 & 28410 & 1276 & 4.07 \\
 50 & 27626 & 1216 & 3.88 \\
100 & 27574 & 1103 & 3.52 \\
150 & 26984 & 1017 & 3.25 \\
200 & 26657 &  983 & 3.14 \\
250 & 25812 &  896 & 2.96 \\
300 & 25006 &  928 & 2.86 \\
350 & 23405 &  835 & 2.66 \\
400 & 20621 &  736 & 2.35 \\
450 & 15123 &  655 & 2.09 \\
475 & 11615 &  649 & 2.07 \\
500 &  7059 &  640 & 2.06 \\
525 &  2627 &  632 & 2.04 \\
550 &   991 &  633 & 2.02 \\
575 &   199 &  645 & 2.02 \\
600 &     3 &  629 & 2.01 \\
625 &     0 &    0 & 0 \\
\bottomrule
\end{tabular}
\end{table}

\begin{figure}
    \centering
    \includegraphics[width=0.9\textwidth]{build/plot_countd42.pdf}
    \caption{Plot der Gesamtzählrate gegen den Druck bei $x_0 = 4,2$ cm, wobei über den starken Abfall ein linearer Fit gelegt ist.}
    \label{fig:plot42}
\end{figure}

Mittels Python 3.7.0 werden die Koeffizienten bestimmt:

\begin{align}
    a_{4,2cm} &= (-2,8 \pm 0,2) \cdot 10^{4} \frac{1}{\symup{cm}}
    b_{4,2cm} &= (6,6 \pm 0,4) \cdot 10^{4}.
\end{align}

Damit kann erneut die mittlere Reichweite bestimmt werden:

\begin{center}
    $R_\text{m} = \frac{c_{4,2} - b_{4,2}}{a_{4,2}} = (1,9 \pm 0,2)$ cm.
\end{center}

Dabei ist $c_{4,2}$ die Hälfte der höchsten Gesamtzählrate aus \autoref{tab:d42cm}. Der Fehler nach  Gauß berechnet sich nach

\begin{equation}
    \Delta R_\text{m} = \sqrt{\bigg(\frac{1}{a_{4,2}} \cdot \Delta b_{4,2} \bigg)^2 + \bigg( \frac{c_{4,2}-b_{4,2}}{a_{4,2}^2} \cdot \Delta a_{4,2} \bigg)^2}.
\end{equation}

Zur Bestimmung der Energie werden die Werte aus \autoref{tab:d42cm} aufgetragen. Über das lineare Gefälle wird ein Fit der Form

\begin{equation}
    E_{4,2}(x) = m_{4,2} x + n_{4,2}
\end{equation}

gelegt. Der entstehende Plot ist in \autoref{fig:plotE24} zu sehen.

\begin{figure}
    \centering
    \includegraphics[width=0.9\textwidth]{build/plot_E24.pdf}
    \caption{Plot der Energien bei $x_0 = 4,2$ cm, wobei über den linearen Teil ein Fit gelegt ist.}
    \label{fig:plotE42}
\end{figure}

In diesem Plot ist aus Gründen der Lesbarkeit und Skalierung der letzte Wert nicht aufgenommen. Bei diesem wurden keine Strahlung mehr gemessen.
Mittels Python 3.7.0 werden die Koeffizienten als

\begin{align}
    m_{4,2} &= (-0,98 \pm 0,04) \frac{\symup{MeV}}{\symup{cm}}\\
    n_{4,2} &= (4,01 \pm 0,05) \symup{MeV}
\end{align}

bestimmt.
Somit wird $E_{4,2}(R_\text{m}) = (2,2 \pm 0,2)$ MeV berechnet. Der zugehörige Fehler berechnet sich nach

\begin{equation}
    \Delta E(R_\text{m}) = \sqrt{\bigg( R_\text{m} \cdot \Delta m_{4,2} \bigg)^2 + \bigg(m_{4,2} \cdot \Delta R_\text{m} \bigg)^2 + \bigg(\Delta n_{4,2} \bigg)^2}.
\end{equation}

Erneut wird zu Vergleichszwecken die Energie nach \eqref{eqn:mittlere-reichweite} berechnet:

\begin{center}
    $E_{4,2,R} = \sqrt[3]{\frac{R_\text{m}^2}{3,1}} = (3,2 \pm 0,2)$ MeV,
\end{center}

Der Fehler berechnet sich wie oben nach \eqref{eqn:fehlerenergie}.


\subsection{Verteilung der Gesamtzählrate}

Die aufgenommen Messwerte bei einer Integrationszeit von $\Delta t = 10$ s sind in \autoref{tab:100werte} zu sehen.

\begin{table}[!htp]
\centering
\caption{Anzahl von Counts bei 10s Integrationszeit.}
\label{tab:100werte}
\begin{tabular}{S[table-format=4.0] S[table-format=4.0] S[table-format=4.0] S[table-format=4.0] S[table-format=4.0] S[table-format=4.0] S[table-format=4.0] S[table-format=4.0] S[table-format=4.0] S[table-format=4.0]}
\toprule
\multicolumn{10}{c}{Counts} \\
\midrule

3749 & 4062 & 3812 & 3971 & 3502 & 3827 & 3879 & 3743 & 4069 & 3811 \\
3701 & 3549 & 3773 & 3980 & 3867 & 4014 & 3728 & 3849 & 3636 & 3688 \\
3884 & 4099 & 3744 & 3871 & 3810 & 3948 & 3702 & 3610 & 3864 & 3839 \\
3939 & 3808 & 3628 & 3962 & 3792 & 3882 & 3678 & 3593 & 3725 & 3794 \\
3635 & 3903 & 3651 & 4070 & 4060 & 3936 & 4055 & 3804 & 3776 & 3725 \\
3865 & 3997 & 3809 & 3694 & 3576 & 3670 & 3607 & 3832 & 3935 & 3770 \\
3602 & 3813 & 3841 & 3897 & 3951 & 3901 & 3976 & 3722 & 3755 & 3806 \\
3911 & 3580 & 4037 & 3805 & 3989 & 3957 & 3951 & 4013 & 3751 & 3932 \\
3848 & 3981 & 3776 & 3972 & 3911 & 3954 & 3637 & 3567 & 3942 & 3760 \\
3904 & 4063 & 3706 & 3697 & 3973 & 3910 & 3733 & 3870 & 3741 & 3856 \\ 

\bottomrule
\end{tabular}
\end{table}

Der Erwartungswert dieser Daten berechnet sich nach

\begin{equation}
    \mu = \sum_{i=1}^{N} x_i p_i,
\end{equation}

wobei $N$ die Anzahl der Messwerte, $x_i$ der jeweilige Wert und $p_i$ die Wahrscheinlichkeit für den jeweiligen Wert ist.
In diesem Fall ist $N = 100$ und $p_i = \dfrac{1}{N}$ für alle Daten.

Somit beträgt $\mu \approx 3827$.
Die Varianz lässt sich mittels $\mu$ wie folgt berechnen:

\begin{equation}
    \sigma^2 = \sum_{i=1}^{N} (x_i - \mu)^2 \cdot p_i \approx 1941.
\end{equation}

Damit lässt sich eine Gaußkurve der Form

\begin{equation}
    f(x | \mu, \sigma^2) = \frac{1}{\sqrt{2 \pi \sigma^2}} \cdot e^{- \frac{(x_i - \mu)^2}{2 \sigma^2}}
\end{equation}

berechnen. Zum Vergleich der Werte mit dieser Kurve werden die Werte in einem Balkendiagramm zusammen mit der Kurve aufgetragen.
Die Kurve wird so normiert, dass ihr Peak genauso hoch ist, wie der höchste Balken. Der so entstehende Plot ist in \autoref{fig:plot_gauss} zu sehen.

\begin{figure}
    \centering
    \includegraphics[width=0.9\textwidth]{build/plot_gauss.pdf}
    \caption{Balkendiagramm der Gesamtzählrate mit einer Balkenbreite von 66 und einer Gaußverteilung.}
    \label{fig:plot_gauss}
\end{figure}

Des Weiteren wird ein Plot erstellt, in dem die Daten mit einer Poissonverteilung verglichen werden.

Eine Poissonverteilung berechnet sich nach

\begin{equation}
    P_\lambda (k) = \frac{\lambda^k}{k!} e^{-\lambda},
\end{equation}

wobei $k \in \mathds{N}$ und einem $\lambda$, das die Varianz und den Erwartungswert beschreibt. 
Da für die Poissonverteilung nur diskrete $k$ benutzt werden können und $k=0$ dem kleinsten Wert entsprechen muss,
muss $\lambda$ entsprechend normiert werden.
Es berechnet sich also über

\begin{equation}
    \lambda = \sum_{i=1}^{N} \bigg( p_i \bigg( x_i - x_\text{min} \bigg) \cdot \frac{1}{L_\text{bin}} \bigg) \approx 4,9.
\end{equation}

Dabei ist $L_\text{bin} = 66$ die Größe der einzelnen Bins.
Der entsprechende Plot ist in \autoref{fig:plot_poisson} zu finden.

\begin{figure}
    \centering
    \includegraphics[width=0.9\textwidth]{build/plot_poisson.pdf}
    \caption{Balkendiagramm der Gesamtzählrate mit einer Balkenbreite von 66 und einer Poissonverteilung.}
    \label{fig:plot_poisson}
\end{figure}